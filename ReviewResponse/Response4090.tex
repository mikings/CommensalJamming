\documentclass[11pt]{amsart}
\usepackage{geometry}                % See geometry.pdf to learn the layout options. There are lots.
\geometry{letterpaper}                   % ... or a4paper or a5paper or ... 
%\geometry{landscape}                % Activate for for rotated page geometry
%\usepackage[parfill]{parskip}    % Activate to begin paragraphs with an empty line rather than an indent
\usepackage{graphicx}
\usepackage{amssymb}
\usepackage{epstopdf}
\usepackage{xcolor}
\DeclareGraphicsRule{.tif}{png}{.png}{`convert #1 `dirname #1`/`basename #1 .tif`.png}

\title{Response to Reviewers: Paper 4090}
\author{M.R. Inggs et al.}
%\date{}                                           % Activate to display a given date or no date

\begin{document}
\maketitle
%\section{}
%\subsection{}

\section{Reviewer 1}
\subsection{
Format:}

 Correctly formatted

\subsection{Importance/Relevance: }
Of sufficient interest

\subsection{Novelty/Originality:}
 Moderately original This is a straightforward paper. Its originality lies in the fact that it is the first paper to explicitly address the question of jamming of commensal radar.
 

\subsection{Technical Correctness: }
Definitely correct

\textcolor{blue}{
No action required.}


\subsection{Experimental Validation: }
Limited but convincing The results in the paper are mostly produced by simulation. However, measurements between sites at the locations of the simulation are provided.
\textcolor{blue}{
No action required.}

\subsection{Clarity of Presentation:}
 Very clear

\subsection{Reference to Prior Work:}
 Excellent references

\subsection{Overall Evaluation:}
 Definite accept

\subsection{General:}

\emph{I'd like to see a discussion of the case where the jammer is not pointed at the receiver, whose location is, after all, unknown. Also, for jamming a PMR, omni azimuth antennas would likely be used, especially for selkf-protection.  Also, I'm curious about the effects of this jamming on the intended users of the FM broadcast, who have an......and the rest of this comment is truncated.}

\textcolor{blue}{
This was investigated in a fuller report of this work. Unfortunately, the paper length limitation means details cannot be presented. However, a comment has been added to the paper giving an overview of the results with jammer misalignment and large reduction in jamming.
}

\section{Reviewer 2}

\subsection{Format:}
 Correctly formatted

\subsection{Importance/Relevance: }

Of broad interest

\subsection{Novelty/Originality:}

 Moderately original

\subsection{Technical Correctness:}
 Probably correct

\subsection{Experimental Validation:}
 Sufficient validation/theoretical paper

\subsection{Clarity of Presentation:}

 Very clear

\subsection{Reference to Prior Work:}

 Excellent references

\subsection{Overall Evaluation:}

 Definite accept

\subsection{General:}

\emph{I am confident the content will add value to both Radar \& EW communities in understanding better the influence jamming could have on CR and it addresses concerns for both communities Well written paper, clear and concise}

\textcolor{blue}{
No action required.}


\section{Reviewer 3}

\subsection{Format: }
Correctly formatted

\subsection{Importance/Relevance:}

 Of sufficient interest

\subsection{Novelty/Originality:}

 Moderately original

\subsection{Technical Correctness:}

 Probably correct

\subsection{Experimental Validation:}

\emph{ Limited but convincing I would just like to see a summary performance graph (1-D) for each of the simulation cases. It would be great to see a Pd vs Jamming power or similar graph for the different cases. The 'flat' infinite persitance plot extraction image is great, but does lack in some regards to reveal real performance.}

\textcolor{blue}{
We agree that the flat persistence plots are somewhat limiting, but we have not discovered a simple way to show the correlation that the human reader adds to the slow time ARD plot. For example, some sort of area based SNR value might work, but, again, the correlation over time will be lost. We find adding colour is quite distracting, as the eye tries to line up similar colours. We propose to leave the results as they stand.}

\textcolor{blue}{We argue that the CFAR output is a proxy for Pd (probability of detection, so the CFAR curves we show are, we believe, a sufficient way to indicate that Pd is high. A multi-level (shades of grey) will be quite distracting.}

\subsection{Clarity of Presentation:}

 Very clear

\subsection{Reference to Prior Work: }

References adequate Is there a reference for the claim ("consitent with actual measurements taken at the site ...") on page 4?

\textcolor{blue}{A citation has been included}

\subsection{Overall Evaluation: }

Definite accept

\subsection{General:}

\emph{Fig 1: Image quality can be improved. Credit for image lacking. Table 1 can be somewhat condensed, re-layout. The info in here is also duplicated NUMEROUS times in the article and should be avoided, e.g. 204.8 kHz is used way too many times in the article. Equations (1) to (3) is too elaborate, especially considering the text prior and post the equations. Page 3, the last sentence just before section C, starting with "The minimum distance ..." should be removed, it is general knowledge. Figure 3 text is poor.  Question on Section 4), page 5: Would self-protection in this sense on to limited by regulations?}


\emph{Figure 1 image quality comment:} 

\textcolor{blue}{We believe there may have been a problem with the reviewer's .pdf reader? The image credits Landsat and Google Earth, and we believe these sources are sufficiently well known not to require a formal citation.}


\emph{Table 1 can be condensed:} 

\textcolor{blue}{We have done this.}


\emph{Duplication of parameters:} 

\textcolor{blue}{All the paragraphs describing the simulation have been carefully editted to remove duplication.}


\emph{Resolution criteria:}  

\textcolor{blue}{We have removed the sentence.}

\emph{Figure 3 text is poor.}

\textcolor{blue}{We increased the size of the image, but possibly the .pdf viewer of the reviewer lacked resolution?}

\emph{Question on Section 4), page 5: Would self-protection in this sense on to limited by regulations?}

\textcolor{blue}{We do not fully understand: the text is mal-formed. We do not believe that regulations would apply in the case of conflict, given that the enemy systems would need to co-exist within their own operational environment. Our point is that it does not take much power to provide self-protection.}

\section{Reviewer 4}
\subsection{Format:} Correctly formatted

\subsection{Importance/Relevance:}
 Of sufficient interest

\subsection{Novelty/Originality:}
 Moderately original

\subsection{Technical Correctness: }
Probably correct

\subsection{Experimental Validation:}
 Sufficient validation/theoretical paper

\subsection{Clarity of Presentation:}
 Very clear

Reference to Prior Work: References adequate References are not extensive, but, as the author states, little work has been done on this topic.

\textcolor{blue}{
No action required.}

\subsection{Overall Evaluation:}
 Definite accept

\end{document}  