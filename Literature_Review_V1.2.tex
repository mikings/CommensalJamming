
%\chapter{CR Literature Review}


\subsection{ECM applied to CR}
With regards to research relating to ECM applied to CR, few studies exist in the open literature. However, numerous organisations, typically of national defence status, are investigating CR countermeasures. 

One of the greatest limiting aspects of jamming and deceiving CR is that the receiver location is unknown. Therefore, to avoid using kilowatts of jamming power, an additional means of intelligence has to be used.

For the detection of CR receivers, surveillance with imaging radar or electro-optic cameras to detect the relatively large antennas is one possibility. The authors are aware of work that optimises the receiver location for given performance requirements. Often the optimised result is unrealistic so DTED and GIS data can be utilised to establish the ``realistic ideal'' location for receiver location \cite{hoyuela2005}.

Furthermore ECM designers will have a detailed understanding of possible CR signal characteristics, channel bandwidth, signal modulation and the processing for target
tracking. Some of the ECM techniques being considered include noise-jamming and coherent jamming.

The aim with noise jamming is to raise the noise floor of the CR system. A modest jamming power could be sufficient to raise the noise above the level of the target being tracked. The advantage of this noise jamming technique is that it covers the CR range and Doppler extent. The disadvantage, on the other hand, is that the noise power is widely spread, so a jammer with high ERP would be needed if the jammer is to be effective. 

Another coherent jamming idea occasionally discussed is based on the application of a Digital RF Memory (DRFM) jamming system. DRFM systems digitise the probe radar waveform and re-transmit it with amplitude and Doppler modulation. The amplitude and Doppler modulation is carefully selected to correspond to plausible synthetic targets that the CR could detect. Since the CR receives a copy of the illuminating signal, coherent integration of the deception signal is possible. In contrast to the noise jamming, coherent jamming requires a lower jammer ERP to generate false targets.

\section{Examples of CR vulnerability}

\begin{itemize}
\item On April 23rd, 1999, the Serbian state television headquarters in Belgrade was destroyed by NATO bombing. 

\item Between March 24th, 1999 and June 10th, 1999, bombings reduced the Serbian broadcast infrastructure. 

\item Broadcast infrastructure also destroyed in Iraq, Afghanistan, Libya, Syria and other places. 

\item Broadcast infrastructure can be destroyed in many ways (asymmetric vulnerability), e.g.: Lightning, Sabotage, Terrorism, War, Cyber attack,
\item elimination of power grid 

\end{itemize}

Bottomline: reliance on just one source of illumination is fragile. Future sensors would need sufficient redundancy to utilise different and multiple sources of illumination (both commensal and active).


\subsection{CR with Active Fallback Component (AFC)}

This work \cite{doh} considers the concept of passive radar in conjunction with an active fallback component. The Active Fallback Component (AFC) could take a number of forms, some of which will be discussed here. The success of any sensor depends on its performance and robustness. To improve robustness in a conflict situation, tactics such as sensor relocation are routinely employed so as to deny the enemy fixed targeting locations. In conflict, it is expedient to assume that fixed broadcast infrastructure would be targeted and destroyed. This would, therefore, render passive radars relying solely on broadcast illuminators of opportunity redundant. In other words, passive radars, assuming they would be deployed, would not be robust against the destruction of broadcast infrastructure, which could occur. This work will propose an active component to be used in conjunction with passive radar which can supplement it under normal conditions, and which can be used as a fallback if primary illumination is unavailable. The active component in essence is an additional means of aiding the survival of the covert sensor infrastructure.

The ECM community is in broad agreement that jamming measures against a CR receiver is challenging because the receiver location is difficult to determine. Bistatic receivers usually have the sanctuary of a relatively large spatial separation from the transmitter. If the receiver location were to be discovered, then jamming the system would become relatively simple. An alternative possibility could be deception jamming to generate false-targets that confuse the CR tracker.

In conflict/theatre situations, a sensor should be as RF silent as possible to reduce the risk of it being targeted. With CRs, the dislocated receiver site is RF silent. However, all transmitters are vulnerable to antiradiation homing and physical attack whether they are co-located with a receiver or not.

In a CR it is easy to overlook the strategic role of the transmitter, which is understandable since the receiver is the only instrument under the direct control of the CR designer. But in a conflict situation it is prudent to assume that broadcast infrastructure could be targeted. In the absence of an adequate number of viable illuminators of opportunity, the sensor could be designed to have the flexibility to fallback to a Low Probability of Intercept (LPI) mode to continue to satisfy (to an extent) the requirement for situational awareness.

For a fallback option, an active component could be deployed. This would have the ability to operate in conjunction with a passive radar, which can supplement it under normal conditions, and which can be used as a source of illumination if primary infrastructure were unavailable.

\subsection{System Implications}

If the AFC were to be co-located with a conventional PBR, then it could diminish the discreteness of the PBR receiver. This possible outcome would have to be balanced against improvements in target detection and localisation performance offered by the active component. It is worth noting, however, that even some modern and sophisticated intercept receivers would have difficulty detecting a low-power signal. Furthermore, if the signal also employed randomising waveforms and which could be properly concealed within an illuminator of opportunity operating band (e.g. FM or DVB), then it may be entirely overlooked, and thus neglected as a threat, by an intercept receiver.










